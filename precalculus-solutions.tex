\documentclass[]{report}
\usepackage{amssymb}
\usepackage{cancel}

\title{Precalculus Solutions}
\author{Michael Rocke}

\begin{document}

\maketitle

\section{Algebra}

\subsection{1.1}

\subsubsection{1}
$\sqrt{9} = 3^2$ - Integer or Natural Number\\
$-\frac{2}{3}$ - Rational \\
$ \frac{51}{3} = 17 $ - Integer or Natural Number \\
$ -10 $ - Integer \\
$ -\frac{\pi}{3} $ - Irrational \\
$ \frac{\sqrt{5}}{2} $ - Irrational \\
$ - \sqrt{4} = -2 $ - Integer \\
$ \frac{5}{1234} $ - Rational 

\subsubsection{2}
$ (3a - b)  - [2a - (a + b)] = (3a - b) - [2a - a - b] = (3a - b) - 2a + a + b = 3a - 2a + a - b + b = 2a $ \\


$ [(a + 3b) -a] - [a - (a - 3b)] =  [3b] -  [3b] = 0 $ \\



$ a - \{2a - [b - (3a - 2b)]\} = a  - \{2a - [3b - 3a]\} = a  - \{5a - 3b\} = a - 5a + 3b = 3b - 4a $

\subsubsection{3}
$ 19 \cdot 179 = 2(10 \cdot 179) - 179 = 2(1790) - 179 = 3580 - 179 = 3401 $ \\
$ 510 \cdot 18 = 2(10 \cdot 510) - (2 \cdot 510) = 10200 - 1020 = 9180 $ \\
$ 302 \cdot 11 = 10\cdot 302 + 302 = 3020 + 302 = 3322 $ \\

\subsubsection{4}

$ 12x - 18y + 30 = 6(2x - 3y + 5)$ \\
$ 8x^2 - 12x^3y - 28x^4z = 4x^2(2 - 3xy -7x^2z)$ \\
$ 9abc + 3a^2b^2c^2 = 3abc(3 + abc) $

\subsubsection{5}

$ \frac{a}{b} - \frac{b}{a} = \frac{a^2}{ab} - \frac{b^2}{ab} = \frac{a^2-b^2}{ab} $\\
$\frac{3}{x - 2}  + \frac{1}{2-x} = \frac{3(2-x) + (x-2)}{(x-2)(2-x)} = \frac{4 - 2x}{(x-2)(2-x)} =\frac{2\cancel{(2 - x)}}{(x-2)\cancel{(2-x)}} = \frac{2}{x-2}$\\


$\frac{1}{1 + \frac{1}{x-1}} = \frac{1}{\frac{x-1}{x-1} + \frac{1}{x-1}} = \frac{1}{\frac{x\cancel{-1+1}}{x-1}} = \frac{x-1}{x}$ \\

using an example of x = 2;
$\frac{1}{2} $ 

using an example of x = 3;
$\frac{2}{3} $ 



$\frac{x}{xy^2} + \frac{y}{x^2y} = \frac{x^3y + xy^3}{x^3y^3} = \frac{x^2 + y^2}{x^2y^2}$ \\
$\frac{4a}{b} + \frac{b}{4a} = \frac{16a^2 + b^2}{4ab}$

\subsubsection{6}

a) $5a^{-3}  =  \frac{5}{a^3}$ \\
b) $(5a)^{-3} = \frac{1}{125a^3}$ \\
c) $21 \cdot 719^3 \cdot 7^{-1} \cdot 3 \cdot 719^{-3} = 21 \cdot \cancel{719^3} \cdot \frac{1}{7} \cdot 3 \cdot \cancel{719^{-3}}  = \frac{27 \cdot 3}{7} = 3^2$ \\
d) 1 \\

\subsubsection{7}
a)  $(a^{n-4}b^4)(ab^{n-1})^4 = (a^{n-4}b^4)(a^4b^{4n -4}) = a^nb^{4n}$ \\
b)  $(4a^3b^{-4})(3a^{-1}b^5) = 12a^2b$ \\
c)  $\frac{x^{14}y^5}{x^4y^{-5}} = x^{10}y^{10}$ \\z
d)  $a^2b^2(a^{-2} + b^{-2}) = b^2 + a^2$ \\
e)  $(x+y)(x^{-1} + y^{-1}) =  (x+y)(\frac{1}{x}+\frac{1}{y}) = (x+y)(\frac{y + x}{xy}) = (\frac{x^2 + 2yx + y^2}{xy})$ \\

Using x = 2, y = 3;
$5 (\frac{1}{2}+\frac{1}{3}) = 5 (\frac{5}{6}) = \frac{25}{6}$

f)  $(\frac{a^2b}{c})^4(\frac{a}{b^2c^3})^2(\frac{c^2}{a^2})^5 = (\frac{a^8b^4}{c^4})(\frac{a^2}{b^4c^6})(\frac{c^{10}}{a^{10}}) = \frac{a^{10}b^4c^{10}}{c^{10}b^4a^{10}} = 1$ \\


\subsubsection{8}
Simplify


a) $\sqrt{49} = 7$


b) $\sqrt{144} = 12$

c) $\sqrt{9  + 16} = \sqrt{25} = 5$

d) $\sqrt{36+ 64} = \sqrt{100} = 10$

e) $\sqrt[3]{27} = 3$

f) $\sqrt[4]{81} = 3$

g) $\sqrt[6]{64} = 2$

h) $\sqrt{.64} = \sqrt{\frac{64}{100}} = \frac{8}{10} = 0.8$

i) $\sqrt{.09} = \frac{3}{10} = .3$

j) $\sqrt{\frac{16}{121}} = \frac{\sqrt{16}}{\sqrt{121}} = \frac{4}{11}$

k) $\sqrt{\frac{225}{400}} = \frac{15}{20} = \frac{3}{4}$

l) $\sqrt[3]{-\frac{1}{27}} = -\frac{1}{3}$

m) $\sqrt[3]{\frac{64}{125}} = \frac{4}{5}$

n) $\sqrt[3]{-1000} = -10$

o) $\sqrt{125} = \sqrt{5 * 5 * 5} = 5\sqrt{5}$

p) $\sqrt{625} = 25$

q) $\sqrt[4]{625} = \sqrt[4]{25 * 25} = \sqrt[4]{5 * 5 * 5 * 5} = 5$

r) $\sqrt{18} = \sqrt{2 * 9} = 3\sqrt{2}$

s) $\sqrt{12} = \sqrt{4 * 3} = 2\sqrt{3}$

t) $\sqrt{2} + \sqrt{8} = \sqrt{2} + 2\sqrt{2} = 3\sqrt{2}$

u) $\sqrt{3} + \sqrt[4]{9} =  \sqrt{3} + \sqrt[4]{\sqrt{3} * \sqrt{3} * \sqrt{3} * \sqrt{3}} = 2\sqrt{3} $

v) $\sqrt[3]{54} + \sqrt[3]{250} = \sqrt[3]{3 * 3 * 3 * 2 } + \sqrt[3]{5 * 5 * 5 *2} = 3\sqrt[3]{2} + 5\sqrt[3]{2}  = 8\sqrt[3]2$

w) $\sqrt[10]{32a^5} = \sqrt[10]{2^5a^5} = \sqrt{2a}$

x) $\sqrt{a^2b^4} = ab^2$

y) $\sqrt[4]{a^5} = \sqrt[4]{a * a * a * a * a} = a \sqrt[4]{a}$

z) $\sqrt{1 - (\frac{\sqrt{3}}{2})^2} = \sqrt{1 - (\frac{3}{4})} = \sqrt{\frac{1}{4}}  = \frac{1}{2}$

\subsubsection{9}
Simplify by rationalizing the denominator


a) $\frac{30}{\sqrt{6}} = \frac{30}{\sqrt{6}} \cdot \frac{\sqrt{6}}{\sqrt{6}} = \frac{30\sqrt{6}}{6} = 5\sqrt{6}$

b) $\frac{\sqrt{6} + 2}{\sqrt{6} - 2} = \frac{\sqrt{6} + 2}{\sqrt{6} - 2} \cdot \frac{\sqrt{6} + 2}{\sqrt{6} + 2} = \frac{6 + 4\sqrt{6} + 4}{6 - 4} = \frac{10 + 4\sqrt{6}}{2} = 5 + 2\sqrt{6}$


c) $\frac{2}{\sqrt{7} + \sqrt{5}} = \frac{2}{\sqrt{7} + \sqrt{5}} \cdot  \frac{\sqrt{7} - \sqrt{5}}{\sqrt{7} - \sqrt{5}} = \sqrt{7} - \sqrt{5}$
	
	
\end{document}

