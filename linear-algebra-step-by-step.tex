\documentclass[]{report}
\usepackage{amsmath}

\title{Solutions for Linear Algebra: Step by step by Kuldeep Singh}
\author{Michael Rocke}

\begin{document}

\maketitle

\tableofcontents

\chapter{Linear Equations and Matrices}
\section {Exercises}

\subsection{1}

1.

a) $x - y -z = 3$ is linear, each variable has a degree of 1
\\
b) $\sqrt{x} + y + z = 6$ is not linear, due to the square root of x. 
\\
c) $\cos{x} + \sin{y} = 1$ is not linear, due to trigonomic functions for x and y
\\
d) $\exp{x + y + z} = 1$  is not linear, due to the exponential function
\\
e) $x - 2y + 5z = \sqrt{3}$ is linear, as the square root is not on one of the variables, but is on a constant value of 3. 
\\
f) $x = -3y$ is linear
\\
g) $x = \frac{-b \pm \sqrt{b^2 - 4ac}}{2a}$, is linear. No mention of y or z on the right hand side and it equates to a constant value.
\\
h) $\pi{}x + y + ez = 5$, is linear. 
\\
i) $\sqrt{2}x + \frac{1}{2}y + z = 0$ is linear, the square root is just a coefficient of x
\\
j) $\sinh{^{-1}}({x}) = \ln|x + \sqrt{x^2 + 1}| $ is not linear due to the trigonomic and logarithmic functions
\\
k) $\frac{\pi}{2}x - \sqrt{2}y + z\sin{\pi} = 0$ is linear. $\sin{\pi}$ is a constant coefficient of z, same goes for y and x. each variable has a degree of 1
\\
l) $x^{2^0} + y^{3^0} + z^{3^0} = 0$ is linear. The trick is each power resolves to 1, as $2^0$ and $3^0$ is 1
\\
m) $y^{\cos{^2}(x) + \sin{^2}(x)} + x - z = 9$. This is linear, trick is to know that ${\cos{^2}(x) + \sin{^2}(x)} $ is a trig identity, resolving to 1
\\

2.

a) 
\begin{align*}
x + y = 2 \tag{1}\\
x - y = 0 \tag{2} \\
\begin{split}
x + y = 2\\
+(x - y =0)\\
\hline
2x = 2\\
\end{split}\\
x = 1\\
y = 1 
\end{align*}

b)
\begin{align*}
2x - 3y = 5\\
x - y = 2 \\
2x-2y = 4\\
\begin{split}
2x - 3y = 5\\
-(2x - 2y = 4)\\
\hline
-y = 1 \\
\end{split}\\
y = -1\\
x = 1
\end{align*}

c)
\begin{align*}
2x - 3y = 35\\
x - y = 2\\
2x - 2y = 4\\
\begin{split}
2x - 3y = 35\\
-(2x - 2y = 4)\\
\hline
-y = 31 \\
\end{split}\\
y = -31\\
x = -29
\end{align*}


d) 
\begin{align*}
5x - 7y = 2\\
9x - 3y = 6\\
3x - y = 2 \\
15x - 5y = 10\\
15x - 21y = 6\\
\begin{split}
15x - 5y = 10\\
-(15x - 21y = 6)\\
\hline
16y = 4 \\
\end{split}\\
4y=1\\
y=\frac{1}{4}\\
3x = 2 + \frac{1}{4} = \frac{9}{4}\\
x = \frac{3}{4}
\end{align*}


e) 
\begin{align*}
\pi{}x - 5y = 2\\
\pi{}x - y = 1\\
\begin{split}
\pi{}x - 5y = 2\\
-(\pi{}x - y = 1)\\
\hline
-4y = 1 \\
\end{split}\\
y=-\frac{1}{4}\\
\pi{}x = 1 + y = \frac{3}{4}\\
x = \frac{3}{4\pi}
\end{align*}

f)
\begin{align*}
ex - ey = 2\\
ex + ey = 0\\
\begin{split}
ex - ey = 2\\
+(ex + ey = 0)\\
\hline
2ex = 2 \\
\end{split}\\
x = \frac{1}{e}\\
y = -\frac{1}{e}
\end{align*}

\end{document}