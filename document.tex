\documentclass{article}

\begin{document}

	\section{Algebra}
	
	\subsection{Power laws}
	
	$ a^xa^y = a^{(x+y)}$

	$(a^x)^y = a^{xy}$	
	
	
	\subsection{Fractions}
	
	$ \frac{a}{m} \cdot \frac{b}{n}  = \frac{ab}{nm}$
	$ \frac{1}{2} \cdot \frac{3}{4} = \frac{3}{8} $
	
	$\frac{a}{m} + \frac{b}{n}  = \frac{an + bm}{mn} $
	$\frac{1}{2} + \frac{3}{4} = \frac{4 + 6}{8} = \frac{10}{8} $
	
	\subsection{Solving linear equations with 3 unknowns}
	
	Given 3 identities with 3 unknowns, first we need to get the problem of two identities with just two unknowns.
	This usually involves adding or subtracting an identity from both sides to cancel one of the unknowns.
	Apply this to get 2 identities, with two unknowns, then apply same procedure to get a single unknown which can be solved.
	Once we have solved this, we can plug this number into the original 3 identities and follow same pattern to complete the other two unknowns
	
		
	\subsection{Solving quadratics}
	$ax^2 + bx + c$
	
	$x = \frac{-b \pm \sqrt{-b^2 + 4ac}}{2a} $
	
	\subsection{Quadratics}
	
	$(x + b)^2 = x^2 + 2xb + b^2 $ \\
	$(x - b)^2 = x^2 - 2xb + b^2 $ \\
	$(x+b)(x-b) = x^2 - b^2 $
	
	
	
	
	
\end{document}