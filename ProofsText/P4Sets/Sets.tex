\documentclass[12 pt]{article}
%%%%%%%%%%%%%%%%%%%%%%%%%%%%
%\topmargin 0.2in
%\textheight 10in
%\voffset -1.25in
%\textwidth 6.2in
%\parindent 0.25in
%\itemindent 0.in
%\leftmargin 0.5in
%\hoffset -0.8in

%\addtolength{\textwidth}{ain}
%\addtolength{\hoffset}{-bin}
%\addtolength{\textheight}{cin}
%\addtolength{\voffset}{cin}
%a=2b

\addtolength{\textwidth}{1.in}
\addtolength{\hoffset}{-.5in}
\addtolength{\textheight}{1.in}
\addtolength{\voffset}{-1.in}


%Packages
\usepackage{graphicx}
\usepackage[mathcal,mathscr]{eucal}
\usepackage{amsfonts}
\usepackage{mathrsfs}
\usepackage{amsmath, amsthm, amssymb,epsfig,amscd,multicol}
\usepackage[shortlabels]{enumitem}
\usepackage{xfrac}
\usepackage{xcolor}
\usepackage{tikz}
\usetikzlibrary{arrows}
\usepackage{float}
\usepackage{caption}
\usepackage{subcaption}
\usepackage{tabu}
\usepackage{arydshln}
%\usepackage{mathptmx}

%New Commands
\newcommand{\R}{\mathbb{R}}
\newcommand{\U}{\mathcal{U}}
\newcommand{\Z}{\mathbb{Z}}
\newcommand{\N}{\mathbb{N}}
\newcommand{\Q}{\mathbb{Q}}
\newcommand{\B}{\mathscr{B}}
\newcommand{\vep}{\varepsilon}
\newcommand{\set}[1]{\left\{#1\right\}}
\newcommand{\card}[1]{\left| #1 \right|}
\newcommand{\an}{\set{a_n}}
\newcommand{\bn}{\set{b_n}}
\newcommand{\Ra}{\Rightarrow}
\newcommand{\ra}{\rightarrow}
\newcommand{\ds}{\displaystyle}
\newcommand{\e}[1]{a \equiv b (\bmod \ #1)}
\newcommand{\psub}{\subset}
\newcommand{\ilim}{\lim_{x \ra \infty}}
\newcommand{\esub}{\subseteq}
\renewcommand{\c}[1]{\overline{#1}}
\newcommand{\mybox}{\tikz[baseline=(current bounding box.center)]{\draw (0,0) rectangle (1cm,1cm);}}

\theoremstyle{definition}
\newtheorem{remark}{Remark}

\theoremstyle{plain}

\newtheoremstyle{mytheorem}% name of the style to be used
  {6pt}% measure of space to leave above the theorem. E.g.: 3pt
  {6pt}% measure of space to leave below the theorem. E.g.: 3pt
  {\itshape}% name of font to use in the body of the theorem
  {0pt}% measure of space to indent
  {\bfseries}% name of head font
  {.}% punctuation between head and body
  {5 pt plus 1pt minus 1pt}% space after theorem head; " " = normal interword space
  {}% Manually specify head
  
 	\theoremstyle{mytheorem}
  	\newtheorem{theorem}{Theorem}%[numbering]
	\newtheorem{lemma}{Lemma}
	\newtheorem{cor}{Corollary}
	\newtheorem{claim}{Claim}

\newtheoremstyle{myexample}% name of the style to be used
  {22pt}% measure of space to leave above the theorem. E.g.: 3pt
  {22pt}% measure of space to leave below the theorem. E.g.: 3pt
  {\normalfont}% name of font to use in the body of the theorem
  {0pt}% measure of space to indent
  {\bfseries}% name of head font
  {.}% punctuation between head and body
  {5 pt plus 1pt minus 1pt}% space after theorem head; " " = normal interword space
  {}% Manually specify head

	\theoremstyle{myexample}
	\newtheorem{example}{Example}%[section]
	
\newtheoremstyle{mydefinition}% name of the style to be used
  {12pt}% measure of space to leave above the theorem. E.g.: 3pt
  {12pt}% measure of space to leave below the theorem. E.g.: 3pt
  {\normalfont}% name of font to use in the body of the theorem
  {0pt}% measure of space to indent
  {\bfseries}% name of head font
  {.}% punctuation between head and body
  {5 pt plus 1pt minus 1pt}% space after theorem head; " " = normal interword space
  {}% Manually specify head

	\theoremstyle{mydefinition}
	\newtheorem{definition}{Definition}





\begin{document}
\pagenumbering{gobble}
\begin{center}
\textbf{P.4 Basic Set Theory}
\end{center}

Sets are arguable the most basic building blocks of mathematics.  It is not possible to discuss writing proofs nor to write proofs without know some basic set theory and notation.
\begin{center}
\fbox{\parbox{5.5in}{Goals:
\begin{itemize}
\item Practice with set-builder notation.
\item Practice finding cardinalities.
\end{itemize}
}}
\end{center}

%Having completed the assigned reading, you should be prepared to work through the following exercises.

\begin{enumerate}
\item Write each of the following sets by listing their elements between braces.	
	\begin{enumerate} \itemsep=.75in
	\item $\set{3x+2 : x \in \Z}$
	\item $\set{x \in \Z : -2 \leq x < 7}$
	\item $\set{x \in \R : x^2 + 5x = -6}$
	\item $\set{5a+2b : a,b \in \Z}$
	\vspace{.75in}
	\end{enumerate} 
	
\item Write each of the following sets in set-builder notation.
	\begin{enumerate} \itemsep=.75in
	\item $\set{3,4,5,6,7,8}$
	\item $\set{ \ldots, -6, -3, 0, 3, 6, 9, 12, 15, \ldots }$
	\vspace{.75in}
	\end{enumerate}
	
\item Describe (in your own words) what the cardinality of set is.

\vspace{.5in}
	
\item Determine the following cardinalities.
	\begin{enumerate} \itemsep=.75in
	\item $\card{\set{2,3,4,5}}$
	\item $\card{\set{2,3,\set{4,5}}}$
	\item $\card{\varnothing}$
	\item $\card{\set{\varnothing}}$
	\item $\card{\set{x \in \Z : x^2-3 \leq 0}}$
	\vspace{.75in}
	\end{enumerate}
	
\item \begin{enumerate}
	\item Suppose that $A$, $B$, and $C$.  What does the notation $A \esub B$ and $A \not\esub C$ mean?  Be explicit.
	\vspace{2in}
	\item List all of the subsets of $A=\set{2,5,7}$.
	
	\vspace{1in}
	
	\end{enumerate}
	\end{enumerate}
	The set of all of the subsets of $A$ is called the \textbf{power set} of $A$ and is denoted $\mathscr{P}(A)$.  Symbolically, $\mathscr{P}(A) = \set{X : X \esub A}$.\\
	\begin{enumerate}[resume]
	\item Let $B = \set{0,\set{1}}$.  Determine $\mathscr{P}(B)$.
	\vspace{1in}
	\end{enumerate}


We'll need to use set notation and the concepts listed here and in sections 1.1 and 1.3 of the text throughout the semester.  If you ever find yourself unsure about set notation, don't hesitate to review these concepts and to ask questions.

\end{document}




