\documentclass[12 pt]{article}
%%%%%%%%%%%%%%%%%%%%%%%%%%%%
%\topmargin 0.2in
%\textheight 10in
%\voffset -1.25in
%\textwidth 6.2in
%\parindent 0.25in
%\itemindent 0.in
%\leftmargin 0.5in
%\hoffset -0.8in

%\addtolength{\textwidth}{ain}
%\addtolength{\hoffset}{-bin}
%\addtolength{\textheight}{cin}
%\addtolength{\voffset}{cin}
%a=2b

\addtolength{\textwidth}{1.in}
\addtolength{\hoffset}{-.5in}
\addtolength{\textheight}{1.in}
\addtolength{\voffset}{-1.in}


%Packages
\usepackage{graphicx}
\usepackage[mathcal,mathscr]{eucal}
\usepackage{amsfonts}
\usepackage{mathrsfs}
\usepackage{amsmath, amsthm, amssymb,epsfig,amscd,multicol}
\usepackage{enumerate}
\usepackage{xcolor}
\usepackage{tikz}
\usetikzlibrary{arrows}
\usepackage{float}
\usepackage{caption}
\usepackage{subcaption}
\usepackage{tabu}
\usepackage{arydshln}
%\usepackage{mathptmx}

%New Commands
\newcommand{\R}{\mathbb{R}}
\newcommand{\U}{\mathcal{U}}
\newcommand{\Z}{\mathbb{Z}}

\theoremstyle{definition}
\newtheorem{remark}{Remark}

\theoremstyle{plain}

\newtheoremstyle{mytheorem}% name of the style to be used
  {22pt}% measure of space to leave above the theorem. E.g.: 3pt
  {22pt}% measure of space to leave below the theorem. E.g.: 3pt
  {\itshape}% name of font to use in the body of the theorem
  {0pt}% measure of space to indent
  {\bfseries}% name of head font
  {.}% punctuation between head and body
  {5 pt plus 1pt minus 1pt}% space after theorem head; " " = normal interword space
  {}% Manually specify head
  
 	\theoremstyle{mytheorem}
  	\newtheorem{theorem}{Theorem}[section]%[numbering]
	\newtheorem{lemma}{Lemma}[section]
	\newtheorem{cor}{Colrollary}[section]

\newtheoremstyle{myexample}% name of the style to be used
  {22pt}% measure of space to leave above the theorem. E.g.: 3pt
  {22pt}% measure of space to leave below the theorem. E.g.: 3pt
  {\normalfont}% name of font to use in the body of the theorem
  {0pt}% measure of space to indent
  {\bfseries}% name of head font
  {.}% punctuation between head and body
  {5 pt plus 1pt minus 1pt}% space after theorem head; " " = normal interword space
  {}% Manually specify head

	\theoremstyle{myexample}
	\newtheorem{example}{Example}[section]
	
\newtheoremstyle{mydefinition}% name of the style to be used
  {22pt}% measure of space to leave above the theorem. E.g.: 3pt
  {22pt}% measure of space to leave below the theorem. E.g.: 3pt
  {\normalfont}% name of font to use in the body of the theorem
  {0pt}% measure of space to indent
  {\bfseries}% name of head font
  {.}% punctuation between head and body
  {5 pt plus 1pt minus 1pt}% space after theorem head; " " = normal interword space
  {}% Manually specify head

	\theoremstyle{mydefinition}
	\newtheorem{definition}{Definition}





\begin{document}
\pagenumbering{gobble}
\begin{center}
\textbf{Quantifiers}
\end{center}

Most mathematical statements of any significance contain quantifiers.  Sometimes they're implied, sometimes their explicit, but either way they're important.

\begin{center}
\fbox{\parbox{5.5in}{Goals:
	\begin{itemize}
	\item Translate statements with quantifiers into logical symbols
	\item Determine the truth value of quantified statements
	\end{itemize}
}}
\end{center}

\begin{enumerate}
\item Translate the following logical statement into English sentences.
	\begin{enumerate} \itemsep2in
	\item $ \forall \ n \in \Z, \ n=2k \ \mbox{for} \ k \in \Z$
	\item $ \forall \ x \in \R, \ x^2 >0$
	\item $ \exists \ n \in \Z, \ 2^n<0$
	\item $ \forall \ n \in \Z, \ \exists m \in \Z, \ n+m=2$
	\item $ \exists \ k \in \Z, \ \forall \ n \in \Z, \ kn=0$
	\item $\exists \ n \in \Z, \ \forall \ m \in \Z, \ n+m = 0$
	\item $\exists \ A \subseteq \R, \ |A|< \infty$
	\item $\forall \ n \in \Z, \ \exists \ A \in \mathscr{P}(\mathbb{N}),  |A|<n$
	\end{enumerate}

\vspace{1in}

\item Determine the truth value of every statement in Exercise 1.
\item Translate the following mathematical statements into symbolic form using the symbols $ \wedge$, $\vee$, $\Rightarrow$, $\Leftrightarrow$, $\exists$, and $\forall$.

	\begin{enumerate} \itemsep3in
	\item If $f$ is a continuous function on the interval $[a,b]$ and $N$ is a number between $f(a)$ and $f(b)$ with $f(a) \neq f(b)$ then there exists $c \in (a,b)$ such that $f(c)=N$. 
	
	\item A function $f$ is continuous on $[a,b]$ if and only if $\lim\limits_{x \rightarrow c} = f(c)$ for all $c \in [a,b]$.
	
	\item The limit of the sequence $a_n$ equals $L$ if and only if for all $\epsilon >0$ there exists $N \in \mathbb{N}$ such that if $n \geq N$ then $|a_n-L| < \epsilon$.
	
	\item The limit of a function $f$ at $x=a$ equals $L$ if and only if for all $\epsilon > 0$ there exists $\delta>0$ such that $|f(x)-L|< \epsilon$ if $|x - a| < \delta$.
	\end{enumerate}

\end{enumerate}

\end{document}




