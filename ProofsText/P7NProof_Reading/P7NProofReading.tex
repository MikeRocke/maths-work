
\chapter{Proof Reading with Examples}

One aspect of writing proofs is learning to read and critique them.  In later classes you will reading and trying to understand proofs of concepts with which you are not particularly familiar.  Here, our goal is to learn to read proofs with an eye for critiquing them.  This will help you to find problems with your own proofs.

\begin{center}
\fbox{\parbox{5.5in}{Goals:
	\begin{itemize}
	\item Learn to read proofs using examples
	\item Read an interpret proofs written by others
	\item Find mistakes in proofs
	\end{itemize}
}}
\end{center}

\section{Reading Proofs}

Reading a mathematical proof is not like reading a passage in a novel or an article online.  Reading a proof is an activity in which the reader must take an active role in the process.  Fortunately, a well-written proof usually tells you exactly what you should be doing.  \\

\noindent \textbf{Directions:}  Consider the following claim whose proof is taken from the text.  Answer the questions as you go.

\begin{claim} If $x$ is an even integer, then $x^2-6x+5$ is odd.
\end{claim}

\begin{mdframed}[backgroundcolor=gray!10!]
\begin{proof}
Suppose $x$ is an even integer.  
\begin{mdframed}[backgroundcolor=white]
\begin{question}
\item  Do as the first sentence tells you to do.  Choose an even integer value to assign to the variable $x$.  Write your choice here and confirm that the claim is true for your choice.
\vspace{.5in}
\end{question}
\end{mdframed}

Then $x=2a$ for some $a \in \Z$, by definition of an even integer. 
\begin{mdframed}[backgroundcolor=white]
\begin{question}[start=2]
\item  If this sentence is true, then you should be able to choose the value of $a$ for your choice of $x$.  Write that value here.
\vspace{.25in}
\end{question}
\end{mdframed}
 So
\begin{align*}
	x^2-6x+5 &= (2a)^2-6(2a)+5\\
	 &= 4a^2-12a+5 \\
	 &= 4a^2-12a+4+1\\
	 &= 2(2a^2-6a+2)+1.
\end{align*}
Therefore we have $x^2-6x+5=2b+1$, where $b=2a^2-6a+2 \in \Z$ by closure. 
\begin{mdframed}[backgroundcolor=white]
\begin{question}[start=3]
\item  This is probably overkill, but double check that $b$ is in fact an integer for your $a$ value.
\vspace{.5in}
\end{question}
\end{mdframed}
 Consequently $x^2-6x+5$ is odd, by definition of an odd number.
\end{proof}
\end{mdframed}

In the above questions you checked the mathematics of a proof using examples.  Of course, examples aren't enough to prove general statements, but they can be helpful guides.  Reading along with a proof using examples not only helps us find potential mathematical errors, but it can also serve as a guide for clear proof \textit{writing}.  \\

\noindent \textbf{Directions:}  Consider the following proof of the same claim.  Once again, follow treat the statements made in the proof like instructions.  Reflect on how elements of the proof are introduced.  Did you have to make changes as you read?

\begin{mdframed}[backgroundcolor=white]
\begin{proof} \openup=1em
Suppose $x$ is an integer.\\
Let $x$ be even.\\
Let $a$ be an integer such that $2a=x$.\\
We know there is an $a$ because $x$ is even.\\
So $x^2=(2a)^2=4a^2$.  Let $a^2=m$ so that $x^2=4m$ so $x^2$ is even.\\
Next, $-6x=-6(2a)=2(-6a)$.  Let $-6a=k$ so that $-6x=2k$ so $-6x$ is even.\\
Finally an even plus an even is even plus and odd is odd.
\vspace{.2in}
\end{proof}
\end{mdframed}

As you read your own proofs and the proofs of others this semester, keep in mind that you should always be able to follow along with them using examples.  If you find that doing so is tedious with a particular proof, then there's a good chance the proof could use some rewriting.

\section{Finding Errors in Proofs}
\noindent The following proofs come from assignments of previous students.  Each proof contains at least one error and most contain several.  However, none of these proofs is totally wrong.  They just need some adjusting.  For each proof, find any errors that you can and suggest a correction.\\


\begin{claim}
If $n \in \Z$, then $n^2+7n+6$ is even.
\end{claim}

\begin{mdframed}[backgroundcolor=gray!10!]
\begin{mdframed}[backgroundcolor=white]
\begin{proof} \openup 2em 
Suppose $n$ is an odd integer.  Thus, $n=2k+1,k\in\Z$.  We substitute for $n$ in $n^2+7n+6$ to get:
\openup -1em
	\begin{align*}
	(2k+1)^2+7(2k+1)+6\\
	4k^2+4k+1+14k+7+6\\
	4k^2+18k+14\\
	2(2k^2+9k)+14\\
	\end{align*}
We let $m=(2k^2+9k+7)$ to get $2m+14$.  We know that $m$ will be some integer, therefore, $n^2+7n+6$ is even.
\end{proof}
\end{mdframed}

\begin{mdframed}[backgroundcolor=white]
\begin{proof} \openup 1em
Suppose $n$ is an even integer.  Thus,$n=2k,k\in\Z$.  We substitute for $n$ in $n^2+7n+6$ to get:
	\begin{align*}
	(2k)^2+7(2k)+6\\
	4k^2+14k+6\\
	2(2k^2+7k)+6
	\end{align*}
We let $m=(2k^2+7k+3)$ to get $2m+6$.  We know that $m$ is some integer, therefore, $n^2+7n+6$ is even.
\end{proof}
\end{mdframed}
\end{mdframed}

\begin{claim}
If $n \in \Z$, then $5n^2+3n+7$ is odd.
\end{claim}

\begin{mdframed}[backgroundcolor=gray!10!]
\begin{mdframed}[backgroundcolor=white]
\begin{proof} \openup 2em{Suppose $n$ is some integer.
\begin{description}
\item {Case 1:} Assume $n$ is an even integer.  By definition of an even integer $n=2k$ for some integer $k$.  Then $5n^2+3n+7=5(2k)^2+7=20k^2+6k+6+1.$  We can then factor out a $2$ so $5n^2+3n+7=2(10k^2+3k+3)+1$.  Let $p = 10k^2+3k+3$ so that $5n^2+3n+7=2p+1$.  Which is the definition of an odd integer, thus $5n^2+3n+7$ is odd.
\item{Case 2:} Assume $n$ is an odd integer.  By definition of an odd integer $n=2k+1$ for some integer $k$.  Then $5n^2+3n+7$ is equal to $5(2k+1)^2+3(2k+1)+7=20k^2+16k+14+1$.  We can factor out a 2 so $5n^2+3n+7=2(10k^2+8k+7)+1$.  Let $p=10k^2+8k+7$ so that $5n^2+3n+7=10k^2+8k+7$.  Which is the definition of an odd integer thus, $5n^2+3n+7$ is odd.
\item{Case 3:} Assume $n$ is equal to zero.  Then $5(0)^2+3(0)+7=7$, by definition $7$ is an odd integer.  Thus, $5n^2+3n+7$ is odd.  
\end{description}
Since all three cases math up then for some integer $n$, $5n^2+3n+7$ is odd.}
\end{proof}
\end{mdframed}
\end{mdframed}

\medskip

\begin{claim}
If $x \in \R$ and $x \notin \set{-1,0,1}$, then $x^3<x$ or $x^3>x$.
\end{claim}

\begin{mdframed}[backgroundcolor=gray!10!]
%\begin{mdframed}[backgroundcolor=white]
%\begin{proof} \openup 1em (By Contrapositive) [If $x \in \R$ and $x = x^3$, then $x \in \set{-1,0,1}$.]  Suppose $x=x^3$, then 
% \[0=x^3-x\]
% \[=x(x^2-1).\]
% The zeros of the equation $0=x(x^2-1)$ are 0, -1, and 1.  Therefore $x \in \set{-1,0,1}$.
%\end{proof}
%\end{mdframed}

\begin{mdframed}[backgroundcolor=white]
\begin{proof} \openup 2em  Suppose that $X \in \R$.  We can say that $x(x+1)(x-1)>0$ or $x(x+1)(x-1)<0$ is true when $x \notin \set{-1,0,1}$.  We can rewrite the inequality $x(x+1)(x-1)>0$ to get $x(x^2-1)>0$ which can be factor to $x^3-x>0$ and finally expressed as $x^3>x$.\\
  We can do the same to the inequality $x(x+1)(x-1)<0$ to get $x(x^2-1)<0$ which can be factored to $x^3-x<0$ and finally expressed as $x^3<x$.\\
  Therefore, if $X \in \R$ \& $x \notin \set{-1,0,1}$, then $x^3>x$ or $x^3<x$.
\end{proof}
\end{mdframed}
\end{mdframed}

\newpage 
\begin{claim}  Suppose $a,b,c,d \in \Z$.  If $a \divides b$ and $c \divides d$, then $ac \divides bd.$
\end{claim}

\begin{mdframed}[backgroundcolor=gray!10!]
\begin{mdframed}[backgroundcolor=white]
\begin{proof} \openup 2em
Suppose $a,b,c,d \in \Z$ so that $a \divides b$ and $c \divides d$.  For $a$ to divide $b$, we need an integer $k$ so that $ak=b$.  For $c$ to divide $d$, we need an integer $\ell$ so that $c\ell=d$.  We can multiple each of these sides to give us $bd=akc\ell$ which can be rewritten as $bd=(ac)(k\ell)$.  We can make $k\ell$ some integer $m$ giving us $bd=(ac)(m)$.  Showing that $ac$ divides $bd$.  Therefore, if $a \divides b$ and $c \divides d$, then $ac \divides bd.$
\end{proof}
\end{mdframed}

\begin{mdframed}[backgroundcolor=white]
\begin{proof} \openup 1.5em
By definition, $a$ divides $b$ if $ak=b$ for $k \in \Z$.  That applies to $c$ dividing $d$, so we have $c\ell=d.$  Therefore, $bd=(ak)(c\ell)$ or $bd=(ac)(k\ell)$ which shows that $ac$ is a multiple of $bd$.
\end{proof}
\end{mdframed}
\end{mdframed}

\medskip

\begin{claim}  Suppose $x,y \in \R$.  If $x<y$, then $\ds x<\frac{x+y}{2} < y.$
\end{claim}

\begin{mdframed}[backgroundcolor=gray!10!]
\begin{mdframed}[backgroundcolor=white]
\begin{proof}  \openup 2em Suppose $x,y \in \R$, then $x<\frac{x+y}{2}$ where $x=\frac{2x}{x} = \frac{x+x}{2}$.  By substituting, $\frac{x+x}{2}<\frac{x+y}{2}$ is equivalent to $\frac{2x}{2}<\frac{x+y}{2}$ in which, $x< \frac{x+y}{2}.$  If $\frac{x+y}{2}<y$ and $y = \frac{2y}{2}=\frac{y+y}{2}$, then $\frac{x+y}{2}<\frac{y+y}{2}$.  After multiplying both sides by 2 we get $x+y<y+y$.  Then, by subtracting $y$ we can see that $x<y$.  Therefore, $x<\frac{x+y}{2}<y$.
\end{proof}
\end{mdframed}

\begin{mdframed}[backgroundcolor=white]
\begin{proof} \openup 2em
Let $x,y\in \R$ and suppose $x<y$.  Multiplying each side by $2$, $x<y = 2x<2y$ and subtracting an $x$ and $y$ from both sides we have $-x-y<-y-x$.  Adding $2x$ to the left side, and $2y$ to the right side of the inequality, $2x-x-y<2y-y-x$ which can be rewritten as $2x<x+y<2y$.  Dividing each expression by $2$ we see that when $x<y$ then $x< \frac{x+y}{2} < y$.
\end{proof}
\end{mdframed}
\end{mdframed}

\medskip

%\begin{claim}
%Suppose $x$ and $y$ are positive real numbers.  If $x<y$, then $x^2<y^2$.
%\end{claim}
%
%\begin{mdframed}[backgroundcolor=gray!10!]
%\begin{mdframed}[backgroundcolor=white]
%\begin{proof}
%\openup 2em {Let $x,y \in \R$.  Suppose $x^2<y^2$.  Subtracting $y^2$ from both sides, we get $x^2-y^2<0$.  This can also be written as $(x+y)(x-y)<0.$  By dividing both sides by $(x+y)$, we get $x<y$.  Therefore, if $x<y$, then $x^2<y^2$.}
%\end{proof}
%\end{mdframed}
%
%\begin{mdframed}[backgroundcolor=white]
%\begin{proof}
%\openup 2em {Let $x<y$.  By subtracting $y$ from both sides we see $x-y<0$.  This can also be written as,
%	\begin{align*}
%	& x^4-y^4<0\\
%	& \frac{=(x^2+y^2)(x^2-y^2)}{(x^2+y^2)} \begin{array}{c} < \\ \ \end{array} \frac{0}{(x^2+y^2)}\\
%	&= x^2-y^2 <0\\
%	&=x^2<y^2
%	\end{align*}
%	Therefore, if $x<y$, then $x^2<y^2$.}
%\end{proof}
%\end{mdframed}
%\end{mdframed}





