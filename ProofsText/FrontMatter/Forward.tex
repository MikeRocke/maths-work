\chapter*{To the Reader}

\noindent \textbf{Why this text?}\\
I never had an introductory proof writing course as an undergraduate student.  Through a stroke of luck (the jury's still out on good or bad), a Discrete Mathematics course transferred and earned me credit for the requirement despite it not really covering proofs.  As a result, I had to learn to write proofs the hard way.  I read a lot of proofs in my textbooks and emulated my instructors when I could.  I think I was a little more willing to dive in and \textit{try} to write proofs than some of classmates, I was a little more willing to be wrong, and I spent more time trying to make sure my proofs were right on my own.\\

I don't exactly recommend this path to anyone.  Even into my graduate school years I was haunted my lack of some fundamental skills.  Simply put, I developed the general technique of proof writing and critical thinking, but I was unaware of many common conventions and approaches.  For instance, many students learn that proving sets $A$ and $B$ are equal is most often done by showing that $A \subseteq B$ and $B \subseteq A$.  While I probably wrote many proofs as an undergraduate that boiled down to this technique, it wasn't until my graduate years that I learned it as an approach that could almost always get me to where I wanted to go.\\

With this text I've tried to deliver what worked in my favor while eliminating the troubles I was faced with.  This inquiry based approach forces students to think of proof writing as something other than a formulaic process.  At the same time, I've provided more structure and instruction than I've seen in other inquiry based approaches to proof writing so that students still get the exposure to common proof writing conventions.  This approach is admittedly slower and more painstaking, but most of my own students come out of the experience with the skills and confidence they need to write proofs on their own.\\

\noindent \textbf{To instructors:}\\
This workbook was developed over a period of several years.  It began as class notes to go along with Dr.~Richard Hammack's \textit{Book of Proof}, but I realized early that students retained the class material the most when they were asked to work with it rather than just recording notes.  Over time activities were added and refined based on student learning and feedback.  It's reached a level at which it can probably serve as a solo text, but it is still intended as a companion to \textit{Book of Proof}.  If you choose to adopt this work, please let me know and I'll pass your message along to Dr.~Hammack as well.\\

In the past these materials were distributed to students in class, but this, invariably, led to problems.  Students often lost their old materials, it was difficult to get materials to students that had been absent, and it was difficult to move forward if the class finished something early.  Having the students purchase a printed version of this text or print it themselves eliminates this problem, although it does unfortunately transfer a small cost to the students. \\  

At Plymouth State University we split the traditional introductory proof writing course into two 3-credit courses.  The first focuses on sets, logic, and relations and the second focuses entirely on proof writing.  Together the two courses place an enormous emphasis on cohort building, presentations, writing, and student self-assessment.  We use Richard Hammack's \textit{Book of Proof} during both semesters and only use this text during the second course.  We generally cover a chapter per day during the course with roughly one third of class meetings devoted entirely to presentations. There are definitely chapters like 1 through 3 that can be combined into a single day.  Keep in mind that students are expected to generate 100\% of the work on their own, so even shorter chapters can take one or two class periods.\\

If you adopt this text, some adaptation is recommended.  I've often referred to myself and colleagues in this work and the name of the course.  I've tried to use \LaTeX commands to do so to make overall changes easier.  You might also want to expand chapters 1 through 3 as they are really only used as a review in our course.  You also may need to add some material to adjust the difficulty level.  Students take our introductory proof writing course in their first year, so mention of topics from higher level calculus, linear algebra, and geometries is mostly omitted.\\

Chapters 4 through 14 cover the fundamental material for proof writing.  Individual chapters may be able to stand on their own, but I wouldn't recommend it without making changes.  Chapters 15 through 19 are all related, so it would be hard to skip around in these chapters.  Otherwise, Chapters 15 through 19 could be potentially be left out all together.\\

A penultimate version of this text will include homework exercises and appendices with former student work to be corrected by students taking the course.  Many of the homework exercises I use come from \textit{Book of Proof} with some additional original claims to be proven.  If you choose to make additions as you adapt this work, you're encouraged to share them with others and me.  I also welcome any feedback you may have and would appreciate you letting me know about any errors.\\




\noindent \textbf{To students:}\\
This is not a textbook.  Textbooks give you a lot of information and examples and then have a few exercises for you to try.  Textbooks are great and there are plenty of textbooks out there that cover all the material you need to know to learn to write proofs.  Unfortunately, students often fool themselves into thinking that they understand a concept because they understood what they read in a textbook.  This is a workbook.  There is information given to you and there are even a few examples, but this work is made up of the exercises you need to be able to write proofs on your own.\\

These materials are meant to be used in class.  For that reason, you will have to bring this work with you to every class.  It's not recommended that you try to work ahead as this will damage the integrity of the in-class activities.  Many students forget to actually read what is presented in this work, leaving themselves baffled as to how to proceed.  Mathematics seems to workout best when we proceed slowly, so don't be in a rush to get to questions and write something down.  Working through this workbook doesn't end when class ends.  It's important that you take time on your own to carefully rewrite proofs from class and do your best to find errors in your work.\\

Young mathematicians tend to be blindsided by their introductory proof writing class.  I've had many students tell me they went into mathematics because they weren't creative and didn't like writing.  Then after 12 years on elementary and high school mathematics they're hit out of nowhere with a writing course.  You have every right to be surprised by this sudden demand for explanation, and you should be prepared to find learning to write proofs to be a frustrating and confusing process.  My advice to you is to be patient, deliberate, and careful as you move forward in learning to write proofs.  It's important to understand that you won't understand everything right away and that you'll often think you know what you're doing when you don't.  Always try to keep in the back of your mind that no one makes as many mathematical mistakes as a mathematician.  Why should a math student be any different?  Learn from your mistake and persevere.    

\begin{flushright}
Dr.~Justin Wright\\
Assistant Professor of Mathematics\\
Plymouth State University\\
\texttt{jpwright1@plymouth.edu}
\end{flushright}





