\chapter{Basic Induction}


\noindent Proof by induction is a technique that allows us to prove a variety of statements that have one thing in common: the ``objects'' the statement refers to are in correspondance with the natural numbers.  Often times we use induction to prove statements about the natural numbers, but the method applies in a surprising variety of situations.\\

Unlike many of our proof techniques, a proof by induction does follow a certain form to which we must adhere.  This form is necessary for both the validity of the method, and for others to be able to understand our proofs.

\begin{center}
\fbox{\parbox{5.5in}{Goals:
\begin{itemize}
\item Follow along with the steps of an induction proof
\item Write a basic induction proof
\end{itemize}
}}
\end{center}

\begin{question}
\item Follow along with the induction proof below and answer the questions as you go.

\begin{claim}
For all $n\in \N$, $\ds 1+2+3+\cdots+n = \frac{n(n+1)}{2}$.
\end{claim}

\begin{qpart}
\item This claim establishes a lot.  If we let $P(n)$ be the statement ``$1+2+3\cdots +n = \frac{n(n+1)}{2}$,'' then our claim becomes $\forall \ n\in \N, P(n)$.  What is the statement $P(1)$?  What about $P(2)$?  Are these statements true?
\vspace{1in}
\item By checking that $P(1)$ is true above, you actually did the first step of the induction proof.  This step is known as the ``Basis'' step.  Checking that $P(2)$ is true is not part of the basis step, but it never hurts to check a few values.
\item The next step of induction is the ``induction step.''  The goal here is to show that $P(n) \Rightarrow P(n+1)$.  That is, we will show that ``If $P(n)$ is true, then $P(n+1)$ is true.''  What do you think is the purpose of the induction step, keeping in mind that we've already established that $P(1)$ is true?
\vspace{2in}
\item We almost always use a direct proof to show that $P(n) \Rightarrow P(n+1)$.  That is, we will suppose that $P(n)$ is true and use it to show that $P(n+1)$ is true.  What is the statement $P(n)$?  What is the statement $P(n+1)$?
\vspace{2in}
\item The proof below shows that $P(n) \Rightarrow P(n+1)$.  Follow along with the proof and justify each step.
\begin{proof}
\openup 1.5em Suppose $\ds 1+2+3+\cdots+n= \frac{n(n+1)}{2}$.  Then 
\begin{align*}
1+2+3+\cdots + n + (n+1) &= (1+2+3+\cdots + n)+(n+1)\\
\ &= \frac{n(n+1)}{2} + (n+1) \\
\ &= \frac{n(n+1)}{2} + \frac{2(n+1)}{2}\\
\ &= \frac{n(n+1)+2(n+1)}{2}\\
\ &= \frac{(n+1)(n+2)}{2}\\
\ &= \frac{(n+1)((n+1)+1)}{2}.
\end{align*}
Therefore, if $1+2+\cdots+n = \frac{n(n+1)}{2}$ then $1+2+\cdots + n + (n+1) = \frac{(n+1)((n+1)+1)}{2}$.
\end{proof}
\item Together, the basis step and the induction step make up a proof by induction.
\end{qpart}
\item The following is a proof of the same claim.  The proof correctly follows the format of a proof by induction, but contains a severe flaw.  See if you can find the mistake.
\begin{claim} 
For all $n\in \N$, $\ds 1+2+3+\cdots+n = \frac{n(n+1)}{2}$.
\end{claim}
\begin{proof}  We proceed by mathematical induction.\\
\noindent \textbf{Basis Step:} Notice that when $n=1$ the statement becomes $1= \frac{1(1+1)}{2}$, which is obviously true.\\

\noindent \textbf{Induction Step:}  Suppose $1+2+3+\cdots+n = \frac{n(n+1)}{2}$.  Then
\begin{align*}
1+2+3+\cdots + n + (n+1) &= \frac{(n+1)((n+1)+1)}{2} \\
(1+2+3+\cdots + n) + (n+1) &=  \frac{(n+1)(n+2)}{2}\\
\frac{n(n+1)}{2} + (n+1) &= \frac{n^2+3n+2}{2}\\
n+1 &= \frac{n^2+3n+2}{2}-\frac{n^2+n)}{2} \\
n+1 &= \frac{n^2+3n+2-(n^2+n)}{2}\\
n+1 &= \frac{2n+2}{2}\\
n+1 &= n+1.
\end{align*}
Since the last statement if obviously true, if $1+2+\cdots+n = \frac{n(n+1)}{2}$ then $1+2+\cdots+n+1 = \frac{(n+1)(n+2)}{2}$.
\end{proof}

\newpage
\item Prove the following claim using mathematical induction.
\begin{claim} If $n \in \N$, then $1+3+5 + \cdots + (2n-1) = n^2.$
\end{claim}

\begin{proof}  We proceed by mathematical induction.\\
\noindent \textbf{Basis Step:}\\
\vspace{1in}

\noindent \textbf{Induction Step:}
\vspace{5.5in}\\
\end{proof}
%\be\item Prove the following claim using mathematical induction.  It should look familiar.
%gin{claim}
%For $n \in \Z$, if $n \geq 0$ then $3 | (n^3-n)$.
%\end{claim}
\end{question}





