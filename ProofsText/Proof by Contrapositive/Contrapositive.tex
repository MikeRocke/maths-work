\documentclass[12 pt]{article}
%%%%%%%%%%%%%%%%%%%%%%%%%%%%
%\topmargin 0.2in
%\textheight 10in
%\voffset -1.25in
%\textwidth 6.2in
%\parindent 0.25in
%\itemindent 0.in
%\leftmargin 0.5in
%\hoffset -0.8in

%\addtolength{\textwidth}{ain}
%\addtolength{\hoffset}{-bin}
%\addtolength{\textheight}{cin}
%\addtolength{\voffset}{cin}
%a=2b

\addtolength{\textwidth}{1.in}
\addtolength{\hoffset}{-.5in}
\addtolength{\textheight}{1.in}
\addtolength{\voffset}{-1.in}


%Packages
\usepackage{graphicx}
\usepackage[mathcal,mathscr]{eucal}
\usepackage{amsfonts}
\usepackage{mathrsfs}
\usepackage{amsmath, amsthm, amssymb,epsfig,amscd,multicol}
\usepackage{enumerate}
\usepackage{xcolor}
\usepackage{tikz}
\usetikzlibrary{arrows}
\usepackage{float}
\usepackage{caption}
\usepackage{subcaption}
\usepackage{tabu}
\usepackage{arydshln}
%\usepackage{mathptmx}

%New Commands
\newcommand{\R}{\mathbb{R}}
\newcommand{\U}{\mathcal{U}}
\newcommand{\Z}{\mathbb{Z}}
\newcommand{\N}{\mathbb{N}}
\newcommand{\set}[1]{\left\{#1\right\}}
\newcommand{\ds}{\displaystyle}

\theoremstyle{definition}
\newtheorem{remark}{Remark}

\theoremstyle{plain}

\newtheoremstyle{mytheorem}% name of the style to be used
  {12pt}% measure of space to leave above the theorem. E.g.: 3pt
  {12pt}% measure of space to leave below the theorem. E.g.: 3pt
  {\itshape}% name of font to use in the body of the theorem
  {0pt}% measure of space to indent
  {\bfseries}% name of head font
  {.}% punctuation between head and body
  {5 pt plus 1pt minus 1pt}% space after theorem head; " " = normal interword space
  {}% Manually specify head
  
 	\theoremstyle{mytheorem}
  	\newtheorem{theorem}{Theorem}[section]%[numbering]
	\newtheorem{lemma}{Lemma}
	\newtheorem{cor}{Corollary}[section]
	\newtheorem{claim}{Claim}

\newtheoremstyle{myexample}% name of the style to be used
  {22pt}% measure of space to leave above the theorem. E.g.: 3pt
  {22pt}% measure of space to leave below the theorem. E.g.: 3pt
  {\normalfont}% name of font to use in the body of the theorem
  {0pt}% measure of space to indent
  {\bfseries}% name of head font
  {.}% punctuation between head and body
  {5 pt plus 1pt minus 1pt}% space after theorem head; " " = normal interword space
  {}% Manually specify head

	\theoremstyle{myexample}
	\newtheorem{example}{Example}[section]
	
\newtheoremstyle{mydefinition}% name of the style to be used
  {12pt}% measure of space to leave above the theorem. E.g.: 3pt
  {12pt}% measure of space to leave below the theorem. E.g.: 3pt
  {\normalfont}% name of font to use in the body of the theorem
  {0pt}% measure of space to indent
  {\bfseries}% name of head font
  {.}% punctuation between head and body
  {5 pt plus 1pt minus 1pt}% space after theorem head; " " = normal interword space
  {}% Manually specify head

	\theoremstyle{mydefinition}
	\newtheorem{definition}{Definition}





\begin{document}
\pagenumbering{gobble}
\begin{center}
\textbf{Proof by Contrapositive}
\end{center}

Proof by contrapositive is a very powerful proof technique.  Technically speaking, any statement that can be proven with a direct proof can be proven with a contrapositive proof and vice versa.  However, it is almost always the case that one proof is significantly easier to write.
\begin{center}
\fbox{\parbox{5.5in}{Goals:
	\begin{itemize}
	\item Write proofs using contrapositive
	\item Determine when it is appropriate to use proof by contrapositive
	\end{itemize}
}}
\end{center}

\begin{enumerate}
\item Prove the following claims using a proof by contrapositive.

\begin{claim}  Suppose $n \in \Z$.  If $n^2$ is even, then $n$ is even.
\end{claim}

\vspace{3in}


\begin{claim}  Suppose $x \in \R$.  If $x^2+5x<0$ then $x<0$.
\end{claim}

\newpage

\begin{claim}  Suppose $a$ and $b$ are integers.  If both $ab$ and $a+b$ are even, then both $a$ and $b$ are even.
\end{claim}
\end{enumerate}

\vspace{7in}

\noindent  You should recall the following definition from your reading.
\begin{definition}  Given integers $a$ and $b$ and an $n \in \N$, we say that $a$ and $b$ are \textbf{congruent modulo n} if (and only if) $n|(a-b)$.  We express this as $ a \equiv b (\bmod \ n)$.  If $a$ and $b$ are not congruent modulo $n$, we write this as $a \not\equiv b (\bmod \ n).$  
\end{definition}

\begin{enumerate}
\setcounter{enumi}{1}
\item Prove the following claim using both direct proof and proof by contrapositive.  Which proof is easier to write?  Which is easier to understand?  To prove the claim, you may apply the following variation of a result known as \textit{Euclid's Lemma}.

\begin{lemma}  Let $a,b,p \in \Z$.  If $p$ is prime and $p | ab$, then $p|a$ or $p|b$.
\end{lemma}

\begin{claim} Let $a,b \in \Z$ and $n \in \N$.  If $a \equiv b (\bmod \ n)$ and $a \equiv c (\bmod \ n)$, then $c \equiv b (\bmod \ n)$.
\end{claim}

\newpage

\item There are no distinct rules about when to use a direct proof versus when to use proof by contrapositive.  However, you may have developed some intuition about where you may want to start.  List some ideas you have of how to decide between direct proof or proof by contrapositive.
\end{enumerate}
\end{document}




