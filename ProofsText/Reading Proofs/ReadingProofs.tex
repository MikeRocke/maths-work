\documentclass[12 pt]{article}
%%%%%%%%%%%%%%%%%%%%%%%%%%%%
%\topmargin 0.2in
%\textheight 10in
%\voffset -1.25in
%\textwidth 6.2in
%\parindent 0.25in
%\itemindent 0.in
%\leftmargin 0.5in
%\hoffset -0.8in

%\addtolength{\textwidth}{ain}
%\addtolength{\hoffset}{-bin}
%\addtolength{\textheight}{cin}
%\addtolength{\voffset}{cin}
%a=2b

\addtolength{\textwidth}{1.in}
\addtolength{\hoffset}{-.5in}
\addtolength{\textheight}{1.in}
\addtolength{\voffset}{-1.in}


%Packages
\usepackage{graphicx}
\usepackage[mathcal,mathscr]{eucal}
\usepackage{amsfonts}
\usepackage{mathrsfs}
\usepackage{amsmath, amsthm, amssymb,epsfig,amscd,multicol}
\usepackage{enumerate}
\usepackage{xcolor}
\usepackage{tikz}
\usetikzlibrary{arrows}
\usepackage{float}
\usepackage{caption}
\usepackage{subcaption}
\usepackage{tabu}
\usepackage{arydshln}
%\usepackage{mathptmx}

%New Commands
\newcommand{\R}{\mathbb{R}}
\newcommand{\U}{\mathcal{U}}
\newcommand{\Z}{\mathbb{Z}}
\newcommand{\N}{\mathbb{N}}
\newcommand{\set}[1]{\left\{#1\right\}}
\newcommand{\ds}{\displaystyle}

\theoremstyle{definition}
\newtheorem{remark}{Remark}

\theoremstyle{plain}

\newtheoremstyle{mytheorem}% name of the style to be used
  {12pt}% measure of space to leave above the theorem. E.g.: 3pt
  {12pt}% measure of space to leave below the theorem. E.g.: 3pt
  {\itshape}% name of font to use in the body of the theorem
  {0pt}% measure of space to indent
  {\bfseries}% name of head font
  {.}% punctuation between head and body
  {5 pt plus 1pt minus 1pt}% space after theorem head; " " = normal interword space
  {}% Manually specify head
  
 	\theoremstyle{mytheorem}
  	\newtheorem{theorem}{Theorem}[section]%[numbering]
	\newtheorem{lemma}{Lemma}[section]
	\newtheorem{cor}{Corollary}[section]
	\newtheorem{claim}{Claim}

\newtheoremstyle{myexample}% name of the style to be used
  {22pt}% measure of space to leave above the theorem. E.g.: 3pt
  {22pt}% measure of space to leave below the theorem. E.g.: 3pt
  {\normalfont}% name of font to use in the body of the theorem
  {0pt}% measure of space to indent
  {\bfseries}% name of head font
  {.}% punctuation between head and body
  {5 pt plus 1pt minus 1pt}% space after theorem head; " " = normal interword space
  {}% Manually specify head

	\theoremstyle{myexample}
	\newtheorem{example}{Example}[section]
	
\newtheoremstyle{mydefinition}% name of the style to be used
  {12pt}% measure of space to leave above the theorem. E.g.: 3pt
  {12pt}% measure of space to leave below the theorem. E.g.: 3pt
  {\normalfont}% name of font to use in the body of the theorem
  {0pt}% measure of space to indent
  {\bfseries}% name of head font
  {.}% punctuation between head and body
  {5 pt plus 1pt minus 1pt}% space after theorem head; " " = normal interword space
  {}% Manually specify head

	\theoremstyle{mydefinition}
	\newtheorem{definition}{Definition}





\begin{document}
\pagenumbering{gobble}
\begin{center}
\textbf{Reading Direct Proofs}
\end{center}

One aspect of writing proofs is learning to read and critique them.  Usually you will reading and trying to understand proofs of concepts with which you are not particularly familiar.  Here, we will read and critique proofs on which we have previously worked.

\begin{center}
\fbox{\parbox{5.5in}{Goals:
	\begin{itemize}
	\item Read an interpret proofs written by others
	\item Find mistakes in proofs
	\end{itemize}
}}
\end{center}

%We will need the following definitions throughout the remainder of the semester.
\noindent You wrote proofs for the following claims for the Week 7 homework.  At least two proofs written by you and your classmates is given for each of the claims from the assignment.  Each proof contains some error or errors.  These errors may be minor, but your job is to find them and suggest a correction.  Recall that the directions said to give a direct proof of the claim.

 \begin{claim}  If $x$ is an odd integer, then $x^3$ is odd.
\end{claim}
\vspace{-.3in}
\begin{proof} \openup 2em {Suppose integer $x$ is odd.  By definition $x=2k+1$ for $k \in \Z$.  Notice $x^3$ can be rewritten as $x^3=(2k+1)^3=8x^3+12k^2+8k+1$.  Factoring out a $2$, we get $x^3 = 2(4k^3+6k^2+4k)+1.$  Now let $l=4k^3+6k^2+4k$.  Plugging in $l$, we get $x^3=2l+1$, which is the definition of an odd integer.}
\end{proof}
\begin{center} \underline{\hspace{\textwidth}} \end{center}
\begin{proof} \openup 2em{
Assume $x$ is an odd integer.  By definition of an odd integer $x=2k+1$ for $k \in \Z$.  Substituting into $x^3=(2k+1)^3=(4k^2+4k+1)(2k+1)=8k^3+8k^2+2k+4k^2+4k+1=8k^3+12k^2+6k+1=2(4k^3+6k^2+3k)+1$.  Have $q= (4k^3+6k^2+3k$ with $q \in \Z$.  Substituting in $q$ we get $x^3 = 2(4k^3+6k^2+3k)+1 = 2q+1$.  Therefore, by definition of an odd, if $x$ is odd, then $x^3$ is odd.}
\end{proof}

\newpage

\begin{claim}  Suppose $a$ is an integer.  If $5 | 2a$, then $5|a$.
\end{claim}
\vspace{-.3in}
\begin{proof}  \openup 2em{Let $a \in \Z$ and $5 |2a$.  By definition there exists a $k \in \Z$ such that $5k=2a$.  Since $a$ is an integer, 2a is even which mean $5k$ must also be even.  Since $5$ is odd, $k$ must be even so $k=2m$, $m \in \Z$.  So $5(2m)=2a$ which equals $10m=2a=5m=a$.  By definition of divides, if $5m=a$ them $5 |a$.}
\end{proof}
\begin{center} \underline{\hspace{\textwidth}} \end{center}
\begin{proof}
\openup 2em {Suppose $5|2a$.  By definition $a|b$ if and only if $ak=b$.  So, $5k=2b$.  Using the laws of evens we find: $5(2m)=2(2a) \Rightarrow 5m=2a$, and we plug in again: $5(2m)=2a \Rightarrow 5m=a.$  Therefore, $5|a$.}
\end{proof}

\begin{center} \underline{\hspace{\textwidth}}\\ \underline{\hspace{\textwidth}} \end{center}

\begin{claim}
If $n \in \Z$, then $5n^2+3n+7$ is odd.
\end{claim}

\begin{proof} \openup 2em{Suppose $n$ is some integer.
\begin{description}
\item {Case 1:} Assume $n$ is an even integer.  By definition of an even integer $n=2k$ for some integer $k$.  Then $5n^2+3n+7=5(2k)^2+7=20k^2+6k+6+1.$  We can then factor out a $2$ so $5n^2+3n+7=2(10k^2+3k+3)+1$.  Let $p = 10k^2+3k+3$ so that $5n^2+3n+7=2p+1$.  Which is the definition of an odd integer, thus $5n^2+3n+7$ is odd.
\item{Case 2:} Assume $n$ is an odd integer.  By definition of an odd integer $n=2k+1$ for some integer $k$.  Then $5n^2+3n+7$ is equal to $5(2k+1)^2+3(2k+1)+7=20k^2+16k+14+1$.  We can factor out a 2 so $5n^2+3n+7=2(10k^2+8k+7)+1$.  Let $p=10k^2+8k+7$ so that $5n^2+3n+7=10k^2+8k+7$.  Which is the definition of an odd integer thus, $5n^2+3n+7$ is odd.
\item{Case 3:} Assume $n$ is equal to zero.  Then $5(0)^2+3(0)+7=7$, by definition $7$ is an odd integer.  Thus, $5n^2+3n+7$ is odd.  
\end{description}
Since all three cases math up then for some integer $n$, $5n^2+3n+7$ is odd.}
\end{proof}

\begin{center} \underline{\hspace{\textwidth}} \end{center}

\begin{proof} \openup 2em{ Let $n$ be an integer.  So $n$ is either even or odd.
\begin{description}
\item{Case 1:} Suppose $n$ is even.  By definition, $n=2k$ for $k \in \Z$.  By substitution,
	\begin{align*}  5(2k)^2+3(2k)+7 &= 5(4k^2)+6k+7\\
		\ &= 20k^2+6k+7\\
		\ &= 2(10k^2+3k+3)+1
	\end{align*}
	Thus, by definition of an odd number, $5n^2+3n+7$ is odd.
\item {Case 2:}  Suppose $n$ is odd.  By definition, $n=2k+1$ for $k \in \Z$.  By substitution, 
	\begin{align*} 5(2k+1)^2+3(2k+1)+7 &= 5(4k^2+4k+1)+6k+3+7\\
	\ &= 20k^2+20k+5+6k+3+7\\
	\ &= 20k^2+26k+15\\
	\ &= 2(10k^2+13k+7)+1
	\end{align*}
	Thus, by definition, $5n^2+3n+7$ is odd.
\end{description}
Therefore, by proof by cases, $5n^2+3n+7$ is odd for all $n \in \Z$.}
\end{proof}

\newpage

\begin{claim}
Suppose $x$ and $y$ are positive real numbers.  If $x<y$, then $x^2<y^2$.
\end{claim}

\begin{proof}
\openup 2em {Let $x,y \in \R$.  Suppose $x^2<y^2$.  Subtracting $y^2$ from both sides, we get $x^2-y^2<0$.  This can also be written as $(x+y)(x-y)<0.$  By dividing both sides by $(x+y)$, we get $x<y$.  Therefore, if $x<y$, then $x^2<y^2$.}
\end{proof}

\begin{center} \underline{\hspace{\textwidth}} \end{center}

\begin{proof}
\openup 2em {Let $x<y$.  By subtracting $y$ from both sides we see $x-y<0$.  This can also be written as,
	\begin{align*}
	& x^4-y^4<0\\
	& \frac{=(x^2+y^2)(x^2-y^2)}{(x^2+y^2)} \begin{array}{c} < \\ \ \end{array} \frac{0}{(x^2+y^2)}\\
	&= x^2-y^2 <0\\
	&=x^2<y^2
	\end{align*}
	Therefore, if $x<y$, then $x^2<y^2$.}
\end{proof}

\begin{center} \underline{\hspace{\textwidth}}\\ \underline{\hspace{\textwidth}} \end{center}


\begin{claim}  If $a$ is an integer and $a^2|a$ then $a \in \set{-1,0,1}$.
\end{claim}

\begin{proof}  \openup 2em {Suppose $a$ is an integer and $a^2$ divides $a$.  By definition of divides $a^2k=a$ for some integer $k$.  We can subtract $a$ so that $a^2k-a=0$ and then factor out an $a$ meaning $a(ak-1)=0$.  Then $a=0$ and $ak-1=0$ then $ak=1$.  The only factors of 1 are 1 and 1 or -1 and -1.  Thus, $a=0$, $a=1$, or $a=-1$.  Therefore, if $a^2$ divides $a$ then $a$ is either $0$, $1$, or $-1$.}
\end{proof}
\begin{center} \underline{\hspace{\textwidth}} \end{center}

\begin{proof}  \openup 2em {Suppose $a$ is an integer and $a^2|a$.  We then know that $a=a^2k$, $k \in \Z$.  
	\begin{description}
	\item{Case 1:}  Suppose $a=0$.  It follows that $a=a^2k=0$ is true for all $k \in \Z$.
	\item{Case 2:}  Suppose $a=-1$.  Then $-1=(-1)^2k$ and $k=-1$, which is an integer.  Therefore, $a^2|a$ when $a=-1$.
	\item{Case 3:}  Suppose $a=1$.  Then $1=(1)^2k$, and $k=1$, which is an integer.  Therefore, $a^2|a$ when $a=1$.
	\item{Case 4:}  Suppose $a \in \Z$ and $a>1$.  Consider that we can express $a =a^2k$ as $k= \frac{1}{a}$.  Notice that $0<k<1$.  As $k \notin \Z$, $a^2 \nmid a$ when $a>1$.
	\item{Case 5:}  Suppose $a \in \Z$ and $a<-1$.  Again consider $k = \frac{1}{a}$.  It follows that $-1<k<0$.  As $k \notin \Z$, $a^2 \nmid a$ when $a<-1$.
	\end{description}
	Thus, we have examined all possibilities of $a \in \Z$ and found that $a^2 | a$ only when $a \in \set{-1,0,1}$.}
\end{proof}
\end{document}




